\section{Security Analysis and Limitations}
\label{sec:discussion}

We analyze the security robustness of \DeepVis against adaptive evasion and discuss operational boundaries.

\noindent\textbf{Robustness against Adaptive Evasion. }
An adversary cognizant of the system might attempt to evade detection by manipulating file attributes.
\begin{itemize}[leftmargin=*]
    \item \textit{Low-Entropy Mimicry:} Padding a malicious binary with null bytes lowers entropy (Red channel evasion). However, this creates a \textit{Trilemma}: padding increases file size or alters structure, triggering Context (Green) or Structure (Blue) alarms. Simultaneous minimization of all three signals while maintaining malicious utility is statistically improbable.
    \item \textit{Hash Collision Targeting:} An attacker might craft filenames to collide with high-churn benign files. \DeepVis mitigates this via Max-Risk Pooling, where the highest risk score dominates the pixel value ($T_{x,y} = \max_i \text{Feature}(f_i)$), preventing signal dilution. Furthermore, assuming the secret key $K$ is protected via ephemeral session generation or privileged memory restrictions, the adversary cannot predict target coordinates.
    \item \textit{Contextual Masking:} Hiding a rootkit in a safe path lowers the Context score but exposes Structural anomalies (e.g., a kernel module in \texttt{/usr/bin}). The feature orthogonality ensures that masking one dimension amplifies anomalies in others.
\end{itemize}

\noindent\textbf{Operational Limitations and Linux-Centric Design. }
\DeepVis prioritizes hyperscale throughput via header-only sampling (first 128 bytes). While this covers 97.1\% of active binary threats (Section~\ref{eval_accuracy}), it inherently misses deep-payload injections in script-based attacks or polyglots. Additionally, our evaluation reveals a performance discrepancy across operating systems: while Linux detection recall is 97.1\%, Windows recall drops to 16.9\%. This is not a structural flaw of the spatial hashing architecture but a consequence of the training data distribution (primarily Linux ELF binaries) and the higher structural variance of Windows PE headers. Future iterations will incorporate Windows-specific feature engineering to address this gap. Currently, \DeepVis functions as a \textit{High-Frequency Triage Filter} for Linux-centric environments, reducing the search space from 100\% of files to 0.6\% of flagged artifacts for deeper forensic analysis.

\noindent\textbf{Resistance to Hash Collisions and FP. }
A key concern in hash-based aggregation is whether collisions between benign files could trigger False Positives (FP). We clarify that \DeepVis is robust against this scenario. Max-Risk Pooling ensures that combining multiple benign files only results in a pixel value representing the riskiest benign file, which by definition remains below the trained anomaly threshold ($\tau$). Unlike summation-based pooling, which accumulates noise, our max-pooling strategy guarantees that colliding legitimate files do not aggregate into a false alarm ($ \max(\text{Benign}_A, \text{Benign}_B) < \tau $). This preserves the low FP rate even under high saturation.

\noindent\textbf{Why Deep Learning over Classical Methods?}
While classical approaches such as One-Class SVMs or threshold-based heuristics offer computational simplicity, they fail to model the non-linear manifold of colliding multi-modal features. In preliminary experiments, SVMs exhibited a 14\% recall degradation under high saturation ($>$500 collisions/pixel) because linear decision boundaries cannot disentangle the max-pooled features of benign files from a malicious signal. The CAE architecture learns to suppress the background noise of benign feature collisions through non-linear channel interactions, representing a structural advantage over linear classifiers. Furthermore, while end-to-end I/O remains $O(N)$ due to physical constraints, our contribution isolates the \textit{verification latency} to $O(1)$ via spatial hashing. This ensures that the detection phase does not become a bottleneck as file counts scale to millions.

\noindent\textbf{Key Rotation and Model Stability.}
Our experiments across 50 independent key rotations show that the threshold $\tau$ remains stable. This stability arises because the CAE learns to reconstruct \textit{per-pixel feature distributions}, which are determined by the underlying file population---not spatial coordinates. Thus, key rotation does not require model retraining.

\noindent\textbf{Deployment and Key Security. }
The integrity of the spatial mapping relies on the secrecy of the HMAC key $K$. In high-security deployments, $K$ should be managed by a Trusted Execution Environment (TEE) or Hardware Security Module (HSM) to prevent host-side extraction. To minimize the Trusted Computing Base (TCB), \DeepVis supports an Agentless Architecture where target snapshots are mounted read-only on a trusted verifier instance.
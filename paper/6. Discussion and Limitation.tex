\section{Security Analysis and Limitations}
\label{sec:discussion}

We analyze the security robustness of \DeepVis against adaptive evasion and discuss operational boundaries.

\noindent\textbf{Robustness against Adaptive Evasion. }
An adversary cognizant of the system might attempt to evade detection by manipulating file attributes.
\begin{itemize}[leftmargin=*]
    \item \textit{Low-Entropy Mimicry:} Padding a malicious binary with null bytes lowers entropy (Red channel evasion). However, this creates a \textit{Trilemma}: padding increases file size or alters structure, triggering Context (Green) or Structure (Blue) alarms. Simultaneous minimization of all three signals while maintaining malicious utility is statistically improbable.
    \item \textit{Hash Collision Targeting:} An attacker might craft filenames to collide with high-churn benign files. \DeepVis mitigates this via Max-Risk Pooling, where the highest risk score dominates the pixel value ($T_{x,y} = \max_i \text{Feature}(f_i)$), preventing signal dilution. Furthermore, assuming the secret key $K$ is protected via ephemeral session generation or privileged memory restrictions, the adversary cannot predict target coordinates.
    \item \textit{Contextual Masking:} Hiding a rootkit in a safe path lowers the Context score but exposes Structural anomalies (e.g., a kernel module in \texttt{/usr/bin}). The feature orthogonality ensures that masking one dimension amplifies anomalies in others.
\end{itemize}

\noindent\textbf{Operational Limitations. }
\DeepVis prioritizes hyperscale throughput via header-only sampling (first 128 bytes). While this covers 96\% of active binary threats (Section~\ref{eval_accuracy}), it inherently misses deep-payload injections in script-based attacks or polyglots. This design reflects a deliberate trade-off: deep-payload scanning at hyperscale is computationally infeasible for continuous monitoring. Thus, \DeepVis functions as a \textit{High-Frequency Triage Filter}, drastically reducing the search space from 100\% of files to 0.6\% of flagged artifacts. This enables heavier forensic tools (e.g., full-content analyzers, memory forensics) to focus exclusively on high-risk targets, trading theoretical completeness for operational feasibility. Additionally, memory-resident threats (e.g., \texttt{ptrace} injections) leave no disk footprint and require complementary memory forensics.

\noindent\textbf{Deployment and Key Security. }
The integrity of the spatial mapping relies on the secrecy of the HMAC key $K$. In high-security deployments, $K$ should be managed by a Trusted Execution Environment (TEE) or Hardware Security Module (HSM) to prevent host-side extraction. To minimize the Trusted Computing Base (TCB), \DeepVis supports an Agentless Architecture where target snapshots are mounted read-only on a trusted verifier instance, isolating the monitoring process from the potentially compromised host kernel.
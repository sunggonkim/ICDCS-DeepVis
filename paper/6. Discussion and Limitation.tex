\section{Security Analysis and Limitations}
\label{sec:discussion}

We analyze the security robustness of \DeepVis against adaptive evasion and discuss operational boundaries.

\noindent\textbf{Robustness against Adaptive Evasion. }
An adversary cognizant of the system might attempt to evade detection by manipulating file attributes.
\begin{itemize}[leftmargin=*]
    \item \textit{Low-Entropy Mimicry:} Padding a malicious binary with null bytes lowers entropy (Red channel evasion). However, this creates a \textit{Trilemma}: padding increases file size or alters structure, triggering Context (Green) or Structure (Blue) alarms. Simultaneous minimization of all three signals while maintaining malicious utility is statistically improbable.
    \item \textit{Hash Collision Targeting:} An attacker might craft filenames to collide with high-churn benign files. \DeepVis mitigates this via Max-Risk Pooling, where the highest risk score dominates the pixel value ($T_{x,y} = \max_i \text{Feature}(f_i)$), preventing signal dilution. Furthermore, assuming the secret key $K$ is protected via ephemeral session generation or privileged memory restrictions, the adversary cannot predict target coordinates.
    \item \textit{Contextual Masking:} Hiding a rootkit in a safe path lowers the Context score but exposes Structural anomalies (e.g., a kernel module in \texttt{/usr/bin}). The feature orthogonality ensures that masking one dimension amplifies anomalies in others.
\end{itemize}

\noindent\textbf{Operational Limitations. }
\DeepVis prioritizes hyperscale throughput via header-only sampling (first 128 bytes). While this covers 97.1\% of active binary threats (Section~\ref{eval_accuracy}), it inherently misses deep-payload injections in script-based attacks or polyglots. This design reflects a deliberate trade-off: deep-payload scanning at hyperscale is computationally infeasible for continuous monitoring. Thus, \DeepVis functions as a \textit{High-Frequency Triage Filter}, drastically reducing the search space from 100\% of files to 0.6\% of flagged artifacts. This enables heavier forensic tools (e.g., full-content analyzers, memory forensics) to focus exclusively on high-risk targets, trading theoretical completeness for operational feasibility.

\noindent\textbf{Statistical Rigor and Calibration.}
To ensure detection reliability across our benign dataset (N=47,270), we employ a 95th percentile calibration strategy for the anomaly threshold $\tau$. This deterministic calibration establishes a conservative baseline that prevents over-fitting to the specific system configuration. While the current 0.3\% FPR is reported at the node/repository level, future work will involve benchmarking against cloud-scale datasets ($>$1M files) across diverse distributions to derive more granular confidence intervals and ROC curves. Currently, \DeepVis achieves its target of providing a low-noise triage signal for security operations environments.

\noindent\textbf{Key Rotation and Threshold Stability.}
A legitimate concern is whether rotating the HMAC key $K$ (which changes the hash-to-coordinate mapping) would invalidate the trained CAE threshold. Our experiments across 50 independent key rotations on GCP show that the threshold $\tau$ remains stable: mean $\tau=0.355$, standard deviation $\sigma<0.001$, with \textbf{100\% Recall maintained across all iterations}. This stability arises because the CAE learns to reconstruct \textit{per-pixel feature distributions} (entropy, context, structure), which are determined by the underlying file population---not the specific spatial coordinates. Thus, key rotation does not require model retraining; only the lookup table for post-hoc attribution must be regenerated.

\noindent\textbf{Scope: Binary Threats, Not Script-Based Attacks.}
\DeepVis is designed as a \textit{First-Stage Triage Filter} for system-level binary threats (kernel rootkits, ELF malware, packed executables), not as a comprehensive antivirus. Script-based attacks (e.g., malicious \texttt{.sh}, \texttt{.py}) and polyglot files fall outside its detection scope, as they often lack the structural anomalies (high entropy, hidden paths) that trigger the RGB channels. These threats are better addressed by complementary runtime monitors (e.g., Falco, OSSEC) that trace execution behavior. By explicitly narrowing the scope to ``deep'' binary threats, \DeepVis achieves the throughput necessary for continuous, fleet-wide monitoring without sacrificing detection fidelity on its target class.

\noindent\textbf{Resistance to Grid Flooding.}
An adversary might attempt to flood the grid by generating a massive number of benign-looking files to saturate pixels and mask malicious signals. \DeepVis is inherently resistant to this attack for two reasons. First, \textit{Key Rotation} ensures the attacker cannot predict which coordinates to target; without knowledge of $K$, flooding is effectively random noise. Second, \textit{Max-Risk Pooling} guarantees that even a single malicious file in a saturated pixel dominates the risk score ($T_{x,y} = \max_i(Feature_i)$). Our saturation experiments (Table~VI) confirm 100\% Recall at 610+ collisions per pixel, validating robustness against intentional flooding.

\noindent\textbf{Deployment and Key Security. }
The integrity of the spatial mapping relies on the secrecy of the HMAC key $K$. In high-security deployments, $K$ should be managed by a Trusted Execution Environment (TEE) or Hardware Security Module (HSM) to prevent host-side extraction. To minimize the Trusted Computing Base (TCB), \DeepVis supports an Agentless Architecture where target snapshots are mounted read-only on a trusted verifier instance, isolating the monitoring process from the potentially compromised host kernel.
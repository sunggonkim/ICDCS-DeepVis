
\section{Limitation and Future work} \label{future}

\noindent\textbf{Uniform Resource Allocation}: Although \DASCALE achieves higher CPU utilization than existing databases, its performance scalability is limited as CPU utilization plateaus beyond a certain point. 
This limitation primarily arises from: (1) reduced per-instance memory allocation with an increasing number of instances, which deteriorates performance in memory-intensive tasks; and (2) the presence of workload skew introduced by heavy transactions that disrupt balanced resource distribution. 
To overcome these issues, adopting workload-aware dynamic pooling of memory and CPU resources appears essential~\cite{oracle2025dynamic, chronis2025memorycentric}. The master node can continuously monitor transaction characteristics and instance states, enabling dynamic reallocation of resources to maintain balance~\cite{oracle2025dynamic}. Furthermore, transaction distribution strategies that consider both workload skew and historical local checkpoint progress (LCP) of slaves may further alleviate resource imbalance~\cite{pavlo2012skew}.


\noindent\textbf{Coordination with Single-Master}: In \DASCALE, the master node coordinates all transactions to ensure concurrency control and data consistency. This centralized coordination eliminates the bottleneck of lock contention, as illustrated in Figure~\ref{fig:lockio_mysql}, by delegating control to a single process. However, this architecture introduces a single point of failure: if the master fails, the entire database system is compromised. To mitigate this risk, adopting a decentralized coordination model coupled with novel concurrency control mechanisms is essential. For instance, leveraging timestamp-based consistency models~\cite{idziorek2023distributed, kang2015spanfs, lam2024accelerating, tanabe2020analysis, 277836, ren2020crossfs} allows for eventual consistency with relaxed synchronization constraints, balancing performance and correctness. Furthermore, exploring multi-master architectures~\cite{depoutovitch2023taurus, li2024gaussdb} can provide resilience and improved resource utilization. In the context of a manycore system, where coordination domains are comparatively small, deploying multiple masters—potentially one per NUMA node—may enhance coordination efficiency and memory locality.


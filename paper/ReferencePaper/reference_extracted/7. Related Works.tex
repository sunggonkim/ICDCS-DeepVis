


\section{Related Works}


\noindent\textbf{Security in HPC.}

\noindent\textbf{Logging in large-scale systems.} With the focus on low-performance overhead and scalability, many logging tools are designed and deployed in many leadership-scale HPCs.
These tools can be categorized into three categories.
1) OS-native system log using \textit{Syslog}~\cite{oliner2007supercomputers, zhang2020efficient}, 2) Focusesing on idivisual system resource or state inlcuding I/O (e.g., Darshan~\cite{snyder2016modular}, Recorder~\cite{wang2020recorder}, Reflector~\cite{al2020reflector}) , Network (e.g., Bro~\cite{sommer2003bro}), and Job scheduling information (e.g., SLURM log~\cite{yoo2003slurm}, PBS log~\cite{pbspro}, and 3) Integrating distributed logs to create comprehensive overview~\cite{kim2020towards, park2017big}.
%As the complexity of HPC systems increases, the logging strategy has evolved to not only focus on the efficient collection of information but also provide a comprehensive understanding of performance degradation. 
Our paper aligns with these studies by utilizing efficiently collected system logs for in-depth analysis. In contrast, we focus on detecting system anomalies through both efficient log collection and analysis. We achieve this by transforming large datasets into compact images that are representative enough to detect system and application anomalies induced by attacks, while maintaining scalability in large HPC systems.


\noindent\textbf{Log-based HPC analysis and optimization.} Many studies have attempted to analyze HPC systems using collected logs and optimize system performance. Other studies focused on utilizing logs to predict performance~\cite{kim2020towards, sung2024a2fl} and analyze the main causes of bottlenecks~\cite{wang2019zoom, yang2024full}. In terms of security, few studies have tried to learn from logs~\cite{demasi2013identifying, nolte2023secure, hickman2018enhancing, prout2016enhancing}. However, these studies mainly focused on static application code analysis~\cite{demasi2013identifying}, encryption-based data security~\cite{nolte2023secure, hickman2018enhancing}, and network breach detection~\cite{prout2016enhancing}. While some works~\cite{luo2020log, cinque2022micro2vec, tan2024maad, wang2022robust} have attempted to apply machine learning-based log anomaly detection in HPC, they either: 1) focused on specific subsets of logs, such as microservice logs~\cite{luo2020log, tan2024maad, huang2025logctbl}, or 2) tested on small subsets or static analyses with test settings and previously accumulated test logs~\cite{cinque2022micro2vec, wang2022robust}.
Our paper aligns with these studies by detecting anomalies in system logs through log analysis. However, our study focuses on the strong high-performance guarantees in HPC systems to provide performance-efficient and scalable log processing and detection through image transformation and anomaly detection, enabling the processing of extremely large, often exascale, HPC logs.






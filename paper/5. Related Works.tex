\section{Related Work}
\label{sec:related_work}

\noindent Scalable Integrity and Cloud-Native Monitoring.
Optimizing system integrity monitoring requires balancing security depth with performance overhead. Traditional File Integrity Monitoring (FIM) tools~\cite{tripwire, aide, samhain} suffer from linear $O(N)$ complexity in hyperscale cloud environments. Provenance-based systems~\cite{unicorn, cheng2024kairos} and runtime monitors~\cite{ossec, falco} incur significant overhead unsuitable for high-frequency verification. Recent eBPF advancements~\cite{he2023cross, lim2024safebpf} reduce monitoring costs but cannot address agent crashes. Kernel subsystems such as IMA and EVM~\cite{zohar2018integrity} utilize Merkle trees~\cite{oprea2007integrity} and TPMs for strict allow-listing, yet face containerization challenges~\cite{luo2019container} and TOCCTOU vulnerabilities~\cite{bohling2020subverting}. In contrast, \DeepVis operates as a scalable yet stateless auditor. Each scan produces a complete snapshot independent of prior state, eliminating consistency gaps while enabling high-frequency verification without kernel reconfiguration, which is critical for cloud environments.


\noindent Malware Visualization and Adversarial Evasion.
Treating binary analysis as a computer vision problem allows systems to bypass the brittleness of signature-based detection. Prior studies~\cite{nataraj2011malware, conti2008visual, aldini2024image} demonstrate that mapping binary files to grayscale images reveals structural patterns distinct to malware families. Entropy analysis~\cite{lyda2007entropy} identifies packed payloads with high information density. However, adversaries employ evasion techniques such as padding or mimicry to fool detectors~\cite{ling2024wolf, uetz2024escape}. \DeepVis addresses these challenges by extending single-file visualization to whole-system tensor representation. Instead of relying on a single metric susceptible to padding, \DeepVis projects the entire filesystem into a fixed-size $128 \times 128 \times 3$ RGB tensor. By encoding entropy (R), context (G), and structure (B), the system leverages feature orthogonality to detect sparse anomalies that evade uni-modal analysis.

\noindent Deep Learning for Anomaly Detection.
Deep learning is widely adopted for detecting anomalies in high-dimensional system data. Approaches such as DeepLog~\cite{du2017deeplog} use LSTM networks to model system logs, while Kitsune~\cite{mirsky2018kitsune} employs autoencoders for network intrusion detection. For high-dimensional tabular or sensor data, unsupervised frameworks such as Deep One-Class Classification~\cite{ruff2018deep}, GANomaly~\cite{akcay2018ganomaly}, and DAGMM~\cite{zong2018deep} learn normal data distributions to flag outliers. Additionally, VAE-based models~\cite{su2019robust, xu2018unsupervised} are effective for multivariate time-series data. However, these methods typically rely on temporal sequences or fixed feature sets. File systems present a unique ordering problem as they are unordered sets of variable-length paths. \DeepVis resolves this by employing a deterministic spatial hash mapping and local max detection ($L_\infty$), enabling the application of convolutional autoencoders to unordered system states without the signal dilution associated with global pooling.


\section{Introduction}
Cloud computing abstracts physical infrastructure into dynamic, ephemeral resources, creating a computational model distinct from traditional on-premise environments.
Ensuring workload integrity across cloud instances and large-scale HPC clusters is critical, as operators must prevent unauthorized modifications across thousands of nodes. 
However, modern applications introduce a tension between security and agility, as frequent deployments and updates cause massive file churn that renders traditional models ineffective.


To address this, two main strategies exist: File Integrity Monitoring (FIM) and Runtime Behavioral Analysis. FIM tools such as AIDE~\cite{aide} and Tripwire~\cite{tripwire} detect static changes through cryptographic hashes, while runtime monitors such as Falco~\cite{falco} and OSSEC~\cite{ossec} trace system calls for anomalies. However, traditional integrity verification faces a fundamental scalability challenge. As the number of files ($N$) grows, the scan latency increases linearly ($O(N)$), causing severe I/O bottlenecks in hyperscale storage. This is problematic because modern cloud instances, despite high CPU throughput, have limited storage bandwidth. For example, scanning a filesystem with millions of small files using synchronous system calls results in excessive context switching and blocking I/O. Beyond the performance cost, the alert fatigue problem further limits usability, as legitimate updates generate thousands of false positives, masking true threats~\cite{arp2022dos}. Thus, the operational cost exceeds the theoretical benefit, forcing operators to disable monitoring during maintenance windows which can create a loophole that attackers can exploit.


\begin{figure}[t]
\centering
\includegraphics[width=0.9\columnwidth]{Figures/fig_motivation_legend.pdf}
\vspace{-1cm} % Restore legend visibility
\subfloat[Scalability]{
    \includegraphics[width=0.47\columnwidth]{Figures/fig_motivation_a.pdf}
}
\hfill
\subfloat[Alert Fatigue]{
    \includegraphics[width=0.47\columnwidth]{Figures/fig_motivation_b.pdf}
}
\vspace{1cm}
\caption{(a) \DeepVis achieves high-throughput saturation via async I/O. (b) Legitimate updates generate false alerts in AIDE.}
\label{fig:motivation}
\end{figure}




Figure~\ref{fig:motivation} compares the scalability and precision of an existing FIM tool (i.e., AIDE) with the proposed scheme (i.e., \DeepVis). We used a cloud instance from a real cloud system as described in Section~\ref{eval_setup} and performed a user directory scan. The AIDE scan time increases linearly with a steep slope, exceeding four minutes for only 100K files. This behavior results from synchronous I/O operations and full file reads required to compute hash values and detect all changes since the latest snapshot. In contrast, \DeepVis reads only the first 4 KB of each file and uses a parallelized asynchronous pipeline to saturate storage bandwidth, keeping scan times under seven seconds for the same dataset. Although file ingestion remains physically \(O(N)\), \DeepVis achieves an effective speedup of up to 121.45$\times$ over traditional full-hash baselines and 16.38$\times$ over modified full-hash schemes that read the same amount of data due to its efficient I/O design. Crucially, \DeepVis ensures that subsequent anomaly detection (inference) time remains constant regardless of dataset size, and alerts only in response to rootkit attacks with 97.1\% precision, whereas AIDE reports every file change, leading to alert fatigue for operators at the expense of full hash verification.

\begin{table}[t]
\caption{Comparison with prior work across four key capabilities: Asynchronous I/O (Async), Obfuscation Resilience (Obfusc.), Zero-Day Detection (0-Day), and Low Overhead (Low Ovhd.).}
\centering
\scriptsize
\begin{tabular}{p{1.3cm}|>{\raggedright\arraybackslash}p{1.5cm}|c|c|c|c}
\toprule
\textbf{Study} & \textbf{Approach} & \textbf{Async} & \textbf{Obfusc.} & \textbf{0-Day} & \textbf{Low Ovhd.} \\
\midrule
AIDE~\cite{aide} & Full-Hash FIM &  & \cmark &  &  \\
Tripwire~\cite{tripwire} & Full-Hash FIM &  & \cmark &  &  \\
ClamAV~\cite{clamav} & Signature Scanning &  &  &  & \cmark \\
Falco~\cite{falco} & Runtime/eBPF & \cmark &  & \cmark &  \\
Unicorn~\cite{unicorn} & Provenance Graph &  & \cmark & \cmark &  \\
OSSEC~\cite{ossec} & Log Analysis &  &  &  & \cmark \\
Set-AE~\cite{zaheer2017deepsets} & Deep Sets Learning & \cmark & \cmark & \cmark & \cmark \\
\hline
\textbf{\DeepVis} & \textbf{Hash-Grid Tensor} & \cmark & \cmark & \cmark & \cmark \\
\bottomrule
\end{tabular}
\label{tab:intro_comparison}
\end{table}


Many previous studies, as summarized in Table~\ref{tab:intro_comparison}, have explored approaches to enhance system monitoring scalability. Traditional FIM tools~\cite{aide, tripwire} prioritize cryptographic exactness via full hashing of entire files but suffer $O(N)$ scalability limits unsuitable for hyperscale systems. Runtime approaches~\cite{falco, unicorn} use eBPF tracing or provenance graphs for zero-day threats but require continuous monitoring that degrades performance. 
Deep learning methods such as Set-AE~\cite{zaheer2017deepsets} provide lightweight set-based anomaly detection models but they can struggle with extereme-scale file corruption scenarios, where single critical modifications dilute within vast benign datasets. 
In contrast to prior sequential scanning approaches, \DeepVis positions itself as a High-Speed Binary Audit tool, distinct from full hash-based FIM. 
It employs asynchronous header-only analysis and maps entire filesystems to fixed-size 2D tensors for CNN processing, enabling scalable cloud deployment with rapid anomaly detection.



This paper proposes \DeepVis, a highly scalable filesystem integrity verification framework for hyperscale distributed systems. Rather than full file hashing, \DeepVis leverages information entropy and heuristic file characteristics including paths, sizes, headers, and extensions to represent them in concise fixed-size tensors for scalability. To this end, \DeepVis (1) implements an asynchronous \texttt{io\_uring} and Rust-based snapshot engine to maximize I/O throughput, (2) transforms file metadata into fixed-size tensors via hash-based partitioning to achieve $O(1)$ inference latency, and (3) utilizes Hash-Grid Parallel CAE with Local Max detection to identify sparse anomalies amid system churn. Positioned as a high-speed binary audit tool, \DeepVis complements full-hash and signature scanners such as AIDE and YARA by detecting structural anomalies in packed binaries and kernel rootkits. Our evaluation using up to 100 real cloud instances demonstrates 121.45$\times$ higher throughput than traditional full-hash baselines and 97.1\% precision for targeted Linux kernel rootkits.



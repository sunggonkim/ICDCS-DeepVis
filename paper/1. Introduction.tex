\section{Introduction}
Cloud computing abstracts physical infrastructure into dynamic, ephemeral resources, creating a computational model distinct from traditional on-premise environments.
Ensuring workload integrity across cloud instances and large-scale HPC clusters is critical, as operators must prevent unauthorized modifications across thousands of nodes. 
However, modern applications introduce a tension between security and agility, as frequent deployments and updates cause massive file churn that renders traditional models ineffective.


Two primary strategies emerge: File Integrity Monitoring (FIM) and Runtime Behavioral Analysis. FIM tools including AIDE~\cite{aide} and Tripwire~\cite{tripwire} detect mutations through cryptographic hashing, while runtime monitors such as Falco~\cite{falco} and OSSEC~\cite{ossec} capture anomalies via system call tracing. Traditional approaches confront fundamental scalability constraints. FIM scan latency exhibits $O(N)$ growth with file count, incurring prohibitive I/O bottlenecks in hyperscale storage where CPU throughput exceeds storage bandwidth. Synchronous filesystem scans across millions of files generate excessive context switching and blocking I/O. Beyond performance, alert fatigue from benign updates obscures genuine threats~\cite{arp2022dos}, forcing operators to suspend monitoring during maintenance windows and creating exploitable security gaps.



\begin{figure}[t]
\centering
\includegraphics[width=0.9\columnwidth]{Figures/fig_motivation_legend.pdf}
\vspace{-1cm} % Restore legend visibility
\subfloat[Scalability]{
    \includegraphics[width=0.47\columnwidth]{Figures/fig_motivation_a.pdf}
}
\hfill
\subfloat[Alert Fatigue]{
    \includegraphics[width=0.47\columnwidth]{Figures/fig_motivation_b.pdf}
}
\vspace{1cm}
\caption{(a) \DeepVis achieves high-throughput saturation via async I/O. (b) Legitimate updates generate false alerts in AIDE.}
\label{fig:motivation}
\end{figure}



Figure~\ref{fig:motivation} contrasts AIDE and \DeepVis performance on a real cloud instance. AIDE exhibits linear scan scaling, requiring four minutes for 100K files due to synchronous I/O and full-file hash computation. \DeepVis reads only the 4KB header per file through parallelized asynchronous I/O, achieving 121.45$\times$ acceleration over AIDE and 16.38$\times$ over modified AIDE with equivalent 4KB read volume. Crucially, \DeepVis anomaly detection latency remains constant independent of dataset size. \DeepVis achieves 97.1\% recall with rootkit-specific alerts, eliminating the alert fatigue inherent in AIDE's exhaustive change reporting.


\begin{table}[t]
\caption{Comparison with prior work across four key capabilities: Asynchronous I/O (Async), Obfuscation Resilience (Obfusc.), Zero-Day Detection (0-Day), and Low Overhead (Low Ovhd.).}
\centering
\scriptsize
\begin{tabular}{p{1.3cm}|>{\raggedright\arraybackslash}p{1.5cm}|c|c|c|c}
\toprule
\textbf{Study} & \textbf{Approach} & \textbf{Async} & \textbf{Obfusc.} & \textbf{0-Day} & \textbf{Low Ovhd.} \\
\midrule
AIDE~\cite{aide} & Full-Hash FIM &  & \cmark &  &  \\
Tripwire~\cite{tripwire} & Full-Hash FIM &  & \cmark &  &  \\
ClamAV~\cite{clamav} & Signature Scanning &  &  &  & \cmark \\
Falco~\cite{falco} & Runtime/eBPF & \cmark &  & \cmark &  \\
Unicorn~\cite{unicorn} & Provenance Graph &  & \cmark & \cmark &  \\
OSSEC~\cite{ossec} & Log Analysis &  &  &  & \cmark \\
Set-AE~\cite{zaheer2017deepsets} & Deep Sets Learning & \cmark & \cmark & \cmark & \cmark \\
\hline
\textbf{\DeepVis} & \textbf{Hash-Grid Tensor} & \cmark & \cmark & \cmark & \cmark \\
\bottomrule
\end{tabular}
\label{tab:intro_comparison}
\end{table}


Many previous studies, as summarized in Table~\ref{tab:intro_comparison}, have explored approaches to enhance system monitoring scalability. Traditional FIM tools~\cite{aide, tripwire} prioritize cryptographic exactness via full-file hashing but suffer $O(N)$ scalability limits unsuitable for hyperscale systems. Runtime approaches~\cite{falco, unicorn} use eBPF tracing or provenance graphs for zero-day threats but require continuous monitoring that degrades performance. Deep learning methods such as Set-AE~\cite{zaheer2017deepsets} provide lightweight set-based anomaly detection but struggle with extreme-scale corruption, where critical modifications dilute within vast benign datasets. In contrast to prior sequential scanning approaches, \DeepVis positions itself as a High-Speed Binary Audit tool, distinct from full hash-based FIM. It employs asynchronous header-only analysis and maps entire filesystems to fixed-size 2D tensors for CNN processing, enabling scalable cloud deployment with rapid anomaly detection.




This paper proposes \DeepVis, a highly scalable filesystem integrity verification framework for hyperscale distributed systems. Rather than full file hashing, \DeepVis leverages information entropy and heuristic file characteristics including paths, sizes, headers, and extensions to represent them in concise fixed-size tensors for scalability. To this end, \DeepVis (1) implements an asynchronous \texttt{io\_uring} and Rust-based snapshot engine to maximize I/O throughput, (2) transforms file metadata into fixed-size tensors via hash-based partitioning to achieve $O(1)$ neural network inference latency, and (3) utilizes Hash-Grid Parallel CAE with Local Max detection to identify sparse anomalies amid system churn. Positioned as a high-speed binary audit tool, \DeepVis complements full-hash and signature scanners such as AIDE and YARA by detecting structural anomalies in packed binaries and kernel rootkits. Our evaluation using up to 100 real cloud instances demonstrates 121.45$\times$ higher throughput than traditional full-hash baselines and 97.1\% recall for targeted Linux kernel rootkits.
